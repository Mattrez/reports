\documentclass{article}
\title{Referat - Systemy operacyjne i administracja}
\author{Mateusz Radomski \and Michał Posłuszny}
\date{gr lab. 8, IT Rok I \\ Numery indeksów: 400569, 401341 }

\usepackage[utf8]{inputenc}
\usepackage[T1]{fontenc}
\usepackage[MeX]{polski}

\begin{document}
\maketitle

\section{Temat - 1}
\subsection{\textit{creat()}}
Wywołanie funkcji \textit{creat()} jest ekwiwalente do wywołania funkcji \textbf{open()} z flagami O\_CREAT, O\_WRONLY, O\_TRUNC.
Otwiera ona plik w trybie zapisu, oraz tworzy nowy plik lub już istniejący przycina do zera.
Zwraca deskryptor pliku, który jest liczbą całkowitą 32 bitową, w języku C jest to typ odpowiadający \textit{int}.

Flaga O\_CREAT tworzy zwykły plik na podanej ścieżce dostępu jeżeli nie istniała ona w chwili wywołania funkcji. Efektem tego jest plik, którego właścicielem jest użytkownik, który stworzył proces wywołujący tę funkcję.

O\_WRONLY oznacza otworzenie pliku tylko w trybie zapisu.

O\_TRUNC informuje funkcję, że jeżeli plik na podanej ścieżce dostępu już istnieje i jest zwykłym plikiem oraz użytkownik ma do niego dostęp, to jest on przycięty do zera.

Funkcja do poprawnego wywołania przyjmuje dwa argumenty ścieżke do otworzenia oraz permisje jakie mają zostać nadane nowo stworzonemu plikowi.

Ścieżka dostępu jest stałym łańcuchem znaków, który jest interpretowany tak jak w środowisku Unix.
Ścieżka zaczynająca się od `/' jest interpretowana jako absolutna, natomiast inne jako relatywne do
ścieżki, w której obecnie znajduje się proces.

Permisje są używane gdy plik w czasie wywołania funkcji nie istniał i zostaje on stworzony.
Przesłane permisje są mu wtedy nadane, permisja taka jest storzona poprzez logiczny operator
\textit{or} w języku C oznaczany symbolem `|'. Należy połączyć dostarczone przez system stałe,
w taki sposób aby nowy plik uzyskał żądane permisje.

Funckja zwraca deskryptor pliku, lub wartość równą $-1$ gdy wystąpił błąd.
W takim przypadku, zmienna globalna \textit{errno} jest ustawiona na wartością opisującą błąd.
\begin{itemize}
\item \textbf{EACCES} - do podanej ścieżki nie mamy dostępu. Zawiera to w sobie również,
iż ścieżka przez, która należałoby przejść aby dostać się do pliku nie zezwala na to.
Również sam plik może nie zezwalać nam na jego otworzenie lub modyfikację.
\item \textbf{EDQUOT} - brak miejsca na dysku dla wywołującego użytkownika.
\item \textbf{EEXIST} - ścieżka już istnieje oraz O\_CREAT i O\_EXCL zostały użyte.
\item \textbf{EFAULT} - łańcuch znaków ścieżki wskazuje na niedostępny adres pamięci.
\item \textbf{EINTR} - sygnał przerwał wykonywanie funkcji.
\item \textbf{EINVAL} - system plików wskazywany przez ścieżke nie wspiera O\_DIRECT;
	niepoprawna flaga lub flaga O\_TMPFILE została przesłana lecz O\_WRONLY ani
	O\_RDWR nie została przesłana.
\item \textbf{EISDIR} - wskazana ścieżka jest już folderem.
\item \textbf{ELOOP} - Podczas przechodzenia przez ścieżkę zbyt wiele linków symbolicznych zostało napotkanych, ścieżka jest linkiem symbolicznym i flaga O\_NOFOLLOW została przesłana
	lecz nie O\_PATH.
\item \textbf{EMFILE} - proces, ma otworzonych zbyt wiele plików.
\item \textbf{ENAMETOOLONG} - ścieżka jest zbyt długa.
\item \textbf{ENFILE} - systemowy limit ilości otworzonych plików został przekroczony.
\item \textbf{ENODEV} - ścieżka wskazuje na plik urządzenia lecz nie istnieje takie urządzenie.
\item \textbf{ENOENT} - część ścieżki dostępu nie istnieje lub ścieżka dostępu jest pusta.
\item \textbf{ENOSPC} - ścieżka miała zostać stworzona lecz, urzędzenie zawierające scieżke ma zbyt mało miejsca na nowy plik.
\item \textbf{ENOTDIR} - ścieżka użyta jako folder nie jest foledrem, lub gdy przesłana została flaga O\_DIRECTORY ściezka jest plikiem.
\item \textbf{ENXIO} - ścieżka jest plikiem urządzenia lecz nie istnieje takie urządzenie lub plik jest \textit{socket'em} UNIX'a.
\item \textbf{EOPNOTSUPP} - system plików zawierający się w ścieżce nie wspiera O\_TMPFILE.
\end{itemize}

Nie istnieją duże różnice pomiędzy tą funkcją na jądrze Linux a na jądrze znajdującym się na systemie Solarisie.
Funkcje odpowiadają temu samemu wywołaniu funkcji \textbf{open()} oraz zachowują się tak samo.

\subsection{\textit{open()}}
Funkcja \textit{open()} potrafi wykonać wiele różnych rzeczy zależnie od podanych jej argumentów.
Lecz jej głównym zadaniem jest otworzenie jakiegoś pliku i zwrócenie do niego deskryptora.

Do funkcji przesyłana jest ścieżka dostępu, ewentualne flagi zmieniające jej działanie oraz możliwe jest przesłanie jej permisji pliku.
Po wywołaniu zwraca wartość taką jak w przypadku funkcji \textbf{creat()}.

Argument ścieżki oraz permisji pliku są identyczne jak w wyżej opisanej funkcji \textbf{creat()}.

Flagi przesłane do funkcji zmieniają wiele rzeczy, najprostsze z nich otwierają plik w innych trybach odczytu/zapisu.
O\_RDONLY, O\_WRONLY, O\_RDWR i O\_APPEND odpowiadają następująco; otworzenie w trybie tylko odczytu, tylko zapisu, odczyt i zapis oraz tryb dodawania do pliku zamiast przyciana go do zera.

\begin{itemize}
\item \textbf{O\_CREAT} - opisane w przykładzie funkcji \textbf{creat()}
\item \textbf{O\_LARGEFILE} - zezwala na używanie przesunięcia w pliku większego niż $2^{32}$
\item \textbf{O\_NONBLOCK} - otwiera plik bez blokowania programu i czekania aż plik się otworzy. Zwracany jest deskryptor pliku lecz nie jest od odrazu gotowy do użycia
\item \textbf{O\_SYNC} - podczas zapisu do pliku proces jest blokowany do momentu dostarczenia danych do zapisywanego sprzętu
\item \textbf{O\_TRUNC} - opisane w przykładzie funkcji \textbf{creat()}
\end{itemize}

Funckja zwraca deskryptor pliku, lub wartość równą $-1$ gdy wystąpił błąd.
W takim przypadku, zmienna globalna \textit{errno} jest ustawiona na wartością opisującą błąd.
Wszystkie błędy są identyczne jak w przyadku funckji \textbf{creat()}.

Funkcja ta na Linux'ie posiada więcej flag jakie można ustawić, jako przykałdy można wymienić: \textbf{O\_ASYNC}, \textbf{O\_DIRECTORY}.
Ponieważ jest więcej flag, które można ustawić, istnieje również więcej błędów na jakie może napotkać funkcja. 

\subsection{\textit{read()}}
Funkcja ta będzie odczytywać $n$ bajtów z przesłanego deskryptora pliku do miejsca w pamięci,
które jest buforem wskazanym przez użytkownika.

Funkcja zostaje wywołana z określonym deskryptorem, adresem, które jest buforem oraz ilością $n$ bajtów, które mają zostać wczytane. Funkcja zwraca wartość typu \textit{ssize\_t}, która określa ilość wczytanych bajtów lub informuje o błędzie.

Deskryptor plików monżna otrzymać poprzez użycie funkcji \textbf{open()} lub \textbf{creat()}.
Bufor, do którego dane będą wczytywane użytkownik musi stworzyć odpowiednio duży, ponieważ funkcja nie sprawdza czy ma możliwość zapisywania do danego miejsca w pamięci.
$n$ bajtów, które zostaną wczytane są liczbą całkowitą typu \textit{size\_t}.
Oznacza to, że maksymalna ilość bajtów jaka może zostać wczytana jest ograniczona przez architekturę bitową systemu operacyjnego, na którym skompilowany został program.

Funckja zwraca ilość wczytanych bajtów, lub wartość równą $-1$ gdy wystąpił błąd.
W takim przypadku, zmienna globalna \textit{errno} jest ustawiona na wartością opisującą błąd.

\begin{itemize}
\item \textbf{EBADF} - z przesłanego deskryptora pliku nie można odczytywać.
\item \textbf{EFAULT} - wskaźnik bufora wskazuje na nielegalne miejsce.
\item \textbf{EINTR} - sygnał przerwał wczytywanie danych przez funkcję.
\item \textbf{EIO} - wystąpił błąd fizyczny z urządzeniem wejścia/wyjścia.
\item \textbf{EISDIR} - przesłany deskryptor jest folderem.
\item \textbf{ENOLINK} - przesłany deskryptor jest linkiem do nieaktywnej maszyny zewnętrznej.
\item \textbf{ESPIPE} - przesłany deskryptor jest powiązany z \textit{pipe}'em.
\end{itemize}

Istnieje różnica pomiędzy wersją tej tej funkcji na systemie Solaris a systemie Linux.
W przypadku wywołania jej z argumentem deskryptora pliku będącego \textit{socket'em} wywołanie to jest równoważne z wywołaniem funkcji \textbf{recv()} bez flag.

  
\section{Temat - 3}
\subsection{\textit{link()}}
Wywołując komendę link tworzymy nowe dowiązanie (zwane też twardym, ang. hard link) najczęściej w innej lokalizacji. Nasza nowa nazwa pliku może być używana dokładnie tak samo jak stara w dowolnych operacjach, ponieważ obie nazwy odnoszą się do tego samego pliku, więc ich właściciel jak i prawa pozostają identyczne, przez co niemożliwe jest wskazania oryginału.

Funkcja jest wywoływana z określonym położeniem pliku i położeniem naszego dowiązania. Po wykonaniu zwraca wartość zero, w przypadku błędu zwraca -1 i ustawia zmienną globalną \textit{errno} na wartość opisującą odpowiedni błąd.


\textbf{Argumentami} tej funkcji są kolejno:
\begin{itemize}
\item \textbf{oldpath} - ścieżka do naszego pliku
\item \textbf{newpath} - ścieżka w której chcemy utworzyć dowiązanie (w momencie w którym w ścieżce newpath plik już istnieje, dowiązanie nie zostanie utworzone a plik nie zostanie nadpisany)
\end{itemize}

\textbf{Opis błędów}
\begin{itemize}
\item \textbf{EACCES} zapis do lokalizacji w której jest \textit{newpath} nie jest dozwolony, bądź \textit{oldpath} lub \textit{newpath} nie daje uprawnień wykonywania (przeszukiwania katalogu).
\item \textbf{EDQUOT} wyczerpano limit bloków dysku użytkownika w tym systemie plików.
\item \textbf{EEXIST} plik \textit{newpath} już istnieje
\item \textbf{EFAULT} \textit{oldpath} lub \textit{newpath} wskazuje na przestrzeń adresową niedostępną dla użytkownika.
\item \textbf{EIO} Wystąpił błąd fizyczny z urządzeniem Wejścia/Wyjścia
\item \textbf{ELOOP} Podczas przetwarzania \textit{oldpath} lub \textit{newpath} napotkano zbyt dużo dowiązań symbolicznych
\item \textbf{EMLINK} Plik do którego odnosi się \textit{oldpath} posiada już maksymalną ilość dowiązań. W systemie \textit{ext4} maksymalna liczba twardych dowiązań to 65,000 
\item \textbf{ENAMETOOLONG} Długość któregoś z argumentów 
przekracza wartość w zmiennej globalnej PATH\_MAX
\item \textbf{ENOENT}  \textit{oldpath} lub \textit{newpath} jest pustą ścieżką, taka ścieżka nie istnieje lub plik  \textit{oldpath} nie istnieje
\item \textbf{ENOLINK}  \textit{oldpath} lub \textit{newpath} wskazuje na zdalną maszynę lub dowiązanie do tej maszyny nie jest już aktywne
\item \textbf{ENOSPC}  Ścieżka w której miało znajdować się dowiązanie nie może być rozszerzone.
\item \textbf{ENOTDIR}  Komponent użyty w jednej z podanych ścieżek nie jest katalogiem.
\item \textbf{EPERM} system plików w których znajdują się \textit{oldpath} i \textit{newpath} nie obsługuje tworzenia dowiązań.
\item \textbf{EROFS} zarządane dowiązanie wymaga zapisu w ścieżce w systemie plików tylko do odczytu.

\item \textbf{EXDEV} \textit{oldpath} i \textit{newpath} leżą w innych systemach plików.


\end{itemize}
\textbf{Różnice między Linuxem a Solarisem}
Nazwy argumentów na Solarisie są inne:
\textit{newpath} nazywa się \textit{new},
\textit{oldpath} nazywa się \textit{existing}
Zamiast błędu \textbf{EIO} w Solarisie możemy napotkać błąd
\textbf{EINTR} Napotkano sygnał podczas wykonywania funkcji link.


\subsection{\textit{symlink()}}

Funkcja symlink tworzy dowiązanie symboliczne (miękkie dowiązanie), o nazwie \textit{linkpath} który zawiera łańcuch znaków pliku \textit{target}. Dowiązania symboliczne mogą zawierać składniki .., które w przypadku użycia na początku dowiązania odnoszą się do katalogów nadrzędnych ścieżki w której utworzono dowiązanie.

Istnieje przypadek w którym dowiązanie symboliczne może wskazywać na nieistniejący plik, jest to tak zwane wiszące dowiązanie.

Prawa dostęp do dowiązania symbolicznego są nieistotne, ignoruje się jego właściciela w trakcie podążania za powiązaniem. Jednakże są one sprawdzane podczas usuwania bądź modyfikacji nazwy dowiązania gdy dowiązanie symboliczne znajduje się w katalogu z ustawionym bitem \textit{sticky} ( \textbf{S-ISVTX} ).

Podczas wywołania podaj się tak jak w przypadku funkcji link() ścieżkę do pliku i ścieżkę w której ma zostać utworzone dowiązanie.
Funkcja zwraca wartość zero gdy się powiedzie, w wypadku błędu zwraca wartość -1 i ustawia zmienną globalną \textit{errno} na wartość opisującą odpowiedni błąd.

\textbf{Argumentami} tej funkcji są kolejno:
\begin{itemize}
\item \textbf{target} - ścieżka do naszego pliku
\item \textbf{linkpath} - ścieżka w której chcemy utworzyć dowiązanie (w momencie w którym w ścieżce newpath plik już istnieje, dowiązanie nie zostanie utworzone a plik nie zostanie nadpisany)
\end{itemize}

\textbf{Opis błędów}
\begin{itemize}
\item \textbf{EACCES} zapis do lokalizacji w której jest \textit{linkpath} nie jest dozwolony, bądź jeden z katalogów w ścieżce \textit{linkpath} nie daje uprawnień wykonywania (przeszukiwania katalogu).
\item \textbf{EDQUOT} wyczerpano limit bloków dysku użytkownika w tym systemie plików.
\item \textbf{EEXIST} plik \textit{linkpath} już istnieje
\item \textbf{EFAULT} \textit{target} lub  \textit{linkpath} wskazuje na przestrzeń adresową niedostępną dla użytkownika.
\item \textbf{EIO} Wystąpił błąd fizyczny z urządzeniem Wejścia/Wyjścia
\item \textbf{ELOOP} Podczas przetwarzania \textit{linkpath} napotkano zbyt dużo dowiązań symbolicznych
\item \textbf{ENAMETOOLONG} Długość któregoś z argumentów 
przekracza wartość w zmiennej globalnej PATH\_MAX
\item \textbf{ENOENT}  \textit{target} lub \textit{linkpath} jest pustą ścieżką, taka ścieżka nie istnieje lub \textit{linkpath} jest wiszącym dowiązaniem symbolicznym. 
\item \textbf{ENOMEM}  Wymagana ilość pamięci jądra nie była dostępna.
\item \textbf{ENOSPC}  Ścieżka w której miało znajdować się dowiązanie nie może być rozszerzone.
\item \textbf{ENOTDIR}  Komponent użyty w jednej z podanych ścieżek nie jest katalogiem.
\item \textbf{EPERM} system plików w których znajduje się \textit{linkpath} nie obsługuje tworzenia symbolicznych dowiązań.
\item \textbf{EROFS} zażądane dowiązanie wymaga zapisu w ścieżce w systemie plików tylko do odczytu.



\end{itemize}
\textbf{Różnice między Linuxem a Solarisem}


Nazwy argumentów na Solarisie są inne:

\textit{linkpath} nazywa się \textit{name1},

\textit{target} nazywa się \textit{name2}

W Solarisie  błąd \textbf{EPERM} nazywa się \textbf{ENOSYS}
Solaris nie posiada błędu informującego o niewystarczającej pamięci jądra \textbf{ENOMEM}


\subsection{\textit{stat()}}

Funkcja\textit{ stat()} zwraca informacje o podanym pliku w buforze wskazanym przez użytkownika. 
Do uzyskania tej informacji nie są wymagane prawa dostępu do samego pliku, lecz wymagane są prawa wykonywania do wszystkich katalogów w ścieżce podanej przez użytkownika. 

Funkcje\textit{ stat()} wywołuje się poprzez podanie ścieżki do pliku i buforu a jej rezultatem jest zwrócenie struktury \textit{stat} do podanego bufora. 

Struktura \textit{stat} wygląda następująco:
\begin{verbatim}
struct stat {
    dev_t     st_dev;      /* ID urządzenia na którym znajduje się plik */
    ino_t     st_ino;      /* numer i-węzła */
    mode_t    st_mode;     /* tryb i typ pliku */
    nlink_t   st_nlink;    /* liczba stałych dowiązań */
    uid_t     st_uid;      /* ID użytkownika właściciela */
    gid_t     st_gid;      /* ID grupy właściciela */
    dev_t     st_rdev;     /* ID urządzenia (dla pliku specjalnego) */
    off_t     st_size;     /* rozmiar pliku w bajtach */
    blksize_t st_blksize;  /* rozmiar bloku dla wyjścia/wejścia systemu pliku */
    blkcnt_t  st_blocks;   /* ilość zaalokowanych bloków o rozmiarach 512-bajtów */
    struct timespec st_atim;    /* czas ostatniego dostępu, jest zmieniany przy procedurach: */
    struct timespec st_mtim;    /* czas ostatniej modyfikacji */
    struct timespec st_ctim;    /* czas ostatniej zmiany informacji i-węzła, np. grupy, liczby dowiązań, praw itp. */
    /* Ta struktura może różnić się w zależności od architektury */
};
\end{verbatim}
Dla pola st\_mode istnieją zdefiniowane maski do typu pliku, są to kolejno:
\begin{description}
\item[S\_IFMT]	0170000	maska bitowa dla pola bitowego typu pliku
\item[S\_IFSOCK]0140000	maska gniazdo
\item[S\_IFLNK]	0120000	maska dowiązania symbolicznego
\item[S\_IFREG]	0100000	maska dla pliku regularnego
\item[S\_IFBLK]	0060000	maska dla urządzenia blokowego
\item[S\_IFDIR]	0040000	maska dla katalogu
\item[S\_IFCHR]	0020000	maska dla urządzenia znakowego
\item[S\_IFIFO]	0010000	maska kolejki First In First Out
\end{description}
Funkcja stat potrafi nam też podać informację o prawach danego pliku poprzez zastosowanie poniższych masek:
\begin{description}

\item[S\_ISUID]	04000	bit "set-used-ID"
\item[S\_ISGID]	02000	bit "set-group-ID" 
\item[S\_ISVTX]	01000	bit "sticky" tylko właściciel może zmieniać nazwę bądź usuwać dany plik lub katalog.
\item[S\_IRWXU]	00700	właściciel ma prawa odczytu, zapisu i wykonania
\item[S\_IRUSR]	00400	właściciel ma prawa odczytu
\item[S\_IWUSR]	00200	właściciel ma prawa zapisu
\item[S\_IXUSR]	00100	właściciel ma prawa wykonania
\item[S\_IRWXG]	00070	grupa ma prawa odczytu, zapisu i wykonania
\item[S\_IRGRP]	00040	grupa ma prawa odczytu
\item[S\_IWGRP]	00020	grupa ma prawa zapisu
\item[S\_IXGRP]	00010	grupa ma prawa wykonania
\item[S\_IRWXO]	00007	pozostali użytkownicy mają prawo odczytu, zapisu i wykonania
\item[S\_IROTH]	00004	pozostali użytkownicy mają prawa odczytu
\item[S\_IWOTH]	00002	pozostali użytkownicy mają prawa zapisu
\item[S\_IXOTH]	00001	pozostali użytkownicy mają prawa wykonania
\end{description}

Funkcja zwraca wartość zero gdy się powiedzie, w wypadku błędu zwraca wartość -1 i ustawia zmienną globalną \textit{errno} na wartość opisującą odpowiedni błąd.

\textbf{Argumentami} tej funkcji są kolejno:
\begin{itemize}
\item \textbf{pathname} - ścieżka do naszego pliku który chcemy przeanalizować
\item \textbf{statbuf} - bufor wskazany przez użytkownika, do którego funkcja przekaże parametry pliku podanego w \textit{pathname}
\end{itemize}

\textbf{Opis błędów}
\begin{itemize}
\item \textbf{EACCES} brakujące uprawnienia wymagane do przeszukania jednego z katalogów w ścieżce \textit{pathname}
\item \textbf{EFAULT} Zły adres.
\item \textbf{ELOOP} Podczas przetwarzania ścieżki \textit{pathname} napotkano zbyt dużo dowiązań symbolicznych
\item \textbf{ENAMETOOLONG} Długość \textit{pathname} jest za długa.
\item \textbf{ENOENT} \textit{pathname} jest pustą ścieżką lub taka ścieżka nie istnieje.

\item \textbf{ENOMEM}  Wymagana ilość pamięci jądra nie była dostępna.
\item \textbf{ENOTDIR}  Komponent użyty w \textit{pathname} nie jest katalogiem.
\item \textbf{EOVERFLOW} \textit{pathname} odnosi się do pliku którego rozmiar, numer i-węzła lub ilość bloków nie może być reprezentowana odpowiedni typami: off\_t, ino\_t, blkcnt\_t. Błąd ten występuje gdy aplikacja skompilowana na platformie 32-bitowej bez \textit{-D\_FILE\_OFFSET\_BITS=64} wywoła \textit{stat()} z użyciem pliku, który jest większy niż (1<<31)-1 bajtów.



\end{itemize}
\textbf{Różnice między Linuxem a Solarisem}


Nazwy argumentów na Solarisie są inne:

\textit{pathname} nazywa się \textit{path},

\textit{statbuf} nazywa się \textit{buf}

W Solarisie  błąd \textbf{EPERM} nazywa się \textbf{ENOSYS}
Struktura stat w systemie Solaris posiada małe różnice względem odpowiednika w systemie Linux.
Kolejność zmiennych jest inna, brakuje także zmiennych czasu ostatniego dostępu, modyfikacji i zmiany statusu.
Zmieniony jest także typ zmiennej \textit{st\_blksize} z \textit{blksize\_t} na \textit{long}.

Jeśli chodzi o kody błędów to Solaris nie ma błędu dotyczącego przekroczenia pamięci jądra - \textbf{ENOMEM}, lecz posiada błąd \textbf{EINTR} - sygnał przerwał wykonywanie funkcji.


























\end{document}
