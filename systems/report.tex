\documentclass{article}
\title{Referat - Systemy operacyjne i administracja}
\author{Mateusz Radomski \and Michał Posłuszny}
\date{gr lab. 8, IT Rok I \\ Numer indeksu: 400569 }

\usepackage[utf8]{inputenc}
\usepackage[T1]{fontenc}
\usepackage[MeX]{polski}

\begin{document}
\maketitle

\section{Temat - 1}
\subsection{\textit{creat()}}
Wywołanie tej funkcji jest ekwiwalente do wywołania funkcji \textbf{open()} z flagami O\_CREAT, O\_WRONLY, O\_TRUNC. Otwiera ona plik w trybie zapisu, oraz tworzy nowy plik lub już istniejący przycina do zera. Zwraca deskryptor pliku, który jest liczbą całkowitą 32 bitową, w języku C jest to typ odpowiadający \textit{int}.

Flaga O\_CREAT tworzy zwykły plik na podanej ścieżce dostępu jeżeli nie istniała ona w chwili wywołania funkcji. Efektem tego jest plik, którego właścicielem jest użytkownik, który stworzył proces wywołujący tę funkcję.

O\_WRONLY oznacza otworzenie pliku tylko w trybie zapisu.

O\_TRUNC informuje funkcję, że jeżeli plik na podanej ścieżce dostępu już istnieje i jest zwykłym plikiem oraz użytkownik ma do niego dostęp, to jest on przycięty do zera.

Funkcja do poprawnego wywołania przyjmuje dwa argumenty ścieżke do otworzenia oraz permisje jakie mają zostać nadane nowo stworzonemu plikowi.

Ścieżka dostępu jest stałym łańcuchem znaków, który jest interpretowany tak jak w środowisku Unix.
Ścieżka zaczynająca się od `/' jest interpretowana jako absolutna, natomiast inne jako relatywne do
ścieżki, w której obecnie znajduje się proces.

Permisje są używane gdy plik w czasie wywołania funkcji nie istniał i zostaje on stworzony.
Przesłane permisje są mu wtedy nadane, permisja taka jest storzona poprzez logiczny operator
\textit{or} w języku C oznaczany symbolem `|'. Należy połączyć dostarczone przez system stałe,
w taki sposób aby nowy plik uzyskał żądane permisje.

Funckja zwraca deskryptor pliku, lub wartość równą $-1$ gdy wystąpił błąd.
W takim przypadku, zmienna globalna \textit{errno} jest ustawiona na wartością opisującą błąd.
\begin{itemize}
\item \textbf{EACCES} - do podanej ścieżki nie mamy dostępu. Zawiera to w sobie również,
iż ścieżka przez, która należałoby przejść aby dostać się do pliku nie zezwala na to.
Również sam plik może nie zezwalać nam na jego otworzenie lub modyfikację.
\item \textbf{EDQUOT} - brak miejsca na dysku dla wywołującego użytkownika.
\item \textbf{EINTR} - sygnał przerwał wykonywanie funkcji.
\item \textbf{EISDIR} - wskazana ścieżka jest już folderem.
\item \textbf{ELOOP} - Podczas przechodzenia przez ścieżkę zbyt wiele linków symbolicznych zostało napotkanych.
\item \textbf{EMFILE} - Proces, ma otworzonych zbyt wiele plików.
\item \textbf{ENOENT} - Cześć ścieżki dostępu nie istnieje lub ścieżka dostępu jest pusta.
\end{itemize}

\subsection{\textit{open()}}
\subsection{\textit{read()}}

\section{Temat - 3}

\end{document}
